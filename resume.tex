\documentclass{article}

\usepackage[a4paper, total={6.5in, 11in}]{geometry}
\usepackage{titlesec}
\usepackage{titling}
\usepackage{fontawesome}
\usepackage{utopia}
\usepackage{hyperref}
\usepackage{graphicx}
\usepackage{multicol}

\titleformat{\section}{\Large}{}{0in}{}[\titlerule]
\titleformat{\subsection}{\large\bfseries}{}{0in}{}
\titleformat{\subsubsection}{\normalsize}{}{0in}{}

\titlespacing{\subsection}{0cm}{0.2cm}{0cm}
\titlespacing{\subsubsection}{0cm}{0cm}{0.5cm}

\input{glyphtounicode}
\pdfgentounicode=1
\pagestyle{empty}
\renewcommand{\labelitemi}{--}
\setlength\parindent{0pt}
\newenvironment{goalfont}{\fontfamily{uncl}\selectfont}{\par}

\begin{document}
\title{Resume}
\author{Mislav Vuletić}

\begin{multicols}{2}
  \fontsize{20}{20}\bfseries
  \theauthor{}

  % \columnbreak
  \fontsize{15}{15}\mdseries
  \mbox{}\\
  \faicon{envelope} mislav.vuletic@gmail.com\\
  \faicon{phone} (+385) 98 9866 039\\
  \faicon{linkedin}/MasterMedo\\
  \faicon{github}/MasterMedo
\end{multicols}

\normalsize

\section{RELEVANT WORK EXPERIENCE}
\subsection{Lead Growth Engineer at Memgraph Ltd. \hfill Mar 2021 --- Present}
\begin{itemize}
  \itemsep0em
  \item developed streaming graph applications at a biweekly cadence, by leading product development, resulting in 4--7\% growth in key product indicators week-over-week
  \item built a distributed music playlist recommendation engine, by using Memgraph, AWS, Kafka, Flask, Redis, Vue and Docker to showcase the power of PageRank and community detection algorithms
  \item analysed the largest cryptocurrency social network, by building a parallelised website crawler and visualising a graph of 1 million nodes, helping investors make smarter decisions
  \item set up company-wide analytics, by creating a number of scripts to gather prospective client data for multi-touch attribution, enabling the company to track users across products
  \item published internal products on official package repositories, by creating and packaging Docker images and PyPI packages, resulting in wider product adoption and superior ease-of-use
  \item {\bfseries tech stack}: Memgraph, Kafka, Redis, Python, Docker, AWS, Cloudflare, Flask, Node.js, React, Vue, Rust, Go
  % \item consolidated all website sub-domains to sub-directories, by utilizing the power of Cloudflare workers, leading to higher search engine rankings and improved control over the website content
\end{itemize}

\subsection{Freelance Software Engineer and Programming Instructor\hfill Feb 2019 --- Mar 2021}
\begin{itemize}
  \itemsep0em
  \item developed multiple RESTful APIs with C\#~.NET Core, Java Spring and Python Flask frameworks
  \item wrote unit, integration and end-to-end CI/CD tests with GitHub, newman, C\# and SQL stored procedures, eliminating manual testing and deployment
  \item taught algorithms and data structures, helping clients stay motivated and accomplish their goals
\end{itemize}

\subsection{Lead Software Engineer at Addiko Bank d.d. \hfill Feb 2019 --- Sept 2019}
\begin{itemize}
  \itemsep0em
  \item created a unified API system to query data warehouses from multiple banks, by designing and developing an object relational mapper, increasing loan approval calculation precision by 5\%
  \item mentored a team of two analyst-turned-software-engineers, by teaching them system design and software engineering principles, helping them accomplish their goals and transition successfully
  \item {\bfseries tech stack}: C\#~.NET Core, SQL, Swagger
\end{itemize}

\section{EDUCATION}
\subsection{Master's degree, Computer Science \hfill Sept 2015 --- July 2020}
University of Zagreb, Faculty of Electrical Engineering and Computing, Croatia\\
Poznań University of Technology, Faculty of Computing, Poland (Student Exchange)

% \subsection{Bachelor's degree, Computer Science \hfill Sept 2015 --- July 2018}
% University of Zagreb, Faculty of Electrical Engineering and Computing, Croatia

% \subsection{High school mathematics graduate \hfill Sept 2011 --- June 2015}
% XV.\ gymnasium (MIOC), Zagreb, Croatia

\section{RELEVANT PERSONAL PROJECTS\hfill featured on \faicon{github}GitHub}
% \subsection{Ticket-tracker\hfill Dec 2020 --- present}
% \begin{itemize}
%   \itemsep0em
%   \item designed and developed a {\bfseries bug tracker} using \textit{flask-sqlalchemy-react} stack
%   \item implemented JWT authentication, url queries, data management
%   \item tags: \textit{flask, sqlalchemy, react, marshmallow, unittest, jwt, restx, migrate, typescript, hooks}
% \end{itemize}

\subsection{Workout repetition counter \hfill Aug 2021 --- present}
\begin{itemize}
  \itemsep0em
  \item designed and built an application for counting exercise repetitions using computer vision
  \item formed and lead a team of software engineers, computer vision researchers, a data analyst and a personal trainer to create a minimum viable product in 48 hours
  \item {\bfseries tech stack}: Tensorflow, OpenCV, Python, pandas, React Native
\end{itemize}

% \subsection{Typetest\hfill Aug 2019 --- present}
% \begin{itemize}
%   \itemsep0em
%   \item built a terminal application for testing, analysing and improving typing speed
%   \item created and published a PyPI package, by collaborating with the open source community to create a superior product, increasing user typing speed by 10\% month-over-month
% \end{itemize}

% \subsection{AI agent for DotA 2\hfill Feb 2020 --- July 2020}
% \begin{itemize}
%   \itemsep0em
%   \item researched DQN, DDQN, TPG and LSTM machine learning techniques
%   \item designed and built a {\bfseries double deep q-learning agent} using \textit{python} and \textit{keras}
%   \item achieved an average of 20 last-hits per game after 1 week of training
% \end{itemize}

% \subsection{Analysing daily expenses \hfill Sept 2018 --- present}
% \begin{itemize}
%   \itemsep0em
%   \item processed daily expenses using \textit{python}; \textit{numpy} and \textit{pandas}
%   \item created insightful infographics with \textit{pyplot}
%   \item tried to predict future expenses using {\bfseries regression analysis} with \textit{scikit-learn}
% \end{itemize}

% \subsection{Temporal texture recognition\hfill Oct 2019 --- Dec 2019}
% \begin{itemize}
%   \itemsep0em
%   \item classified temporal and spatial movement using optical flow with \textit{python} and \textit{cv}
%   \item researched motion detection techniques
% \end{itemize}

\section{TECHNICAL SKILLS}
\begin{itemize}
  \itemsep0em
  \item bilingual English, limited working German language skills
  \item proficient Python knowledge, acquired over 10 years of daily development
  \item practical Linux experience, gathered by using Arch Linux as the primary operating system for 6 years
  \item advanced experience with relational and graph databases, gathered through building object mappers and query builders with SQL and Cypher
  \item product release experience, acquired through publishing and maintaining Docker Hub images, PyPI packages, and a Google Marketplace add-on
\end{itemize}

% \section{COMPETITIONS}
% \begin{itemize}
%   \itemsep0em
%   \item Advent of Code --- reached {\bfseries top 100} several times
%   \item Memgraph Hackathon 2021 --- {\bfseries 1st}, best graph algorithm
%   \item Google HashCode 2020 --- {\bfseries 1st} in cluster, top 3 in Croatia
%   \item Croz CodeQuest 2017 --- {\bfseries 3rd} and 7th place
% \end{itemize}

% \section{ABOUT ME}
% I am a long time {\bfseries Vim} and {\bfseries Arch Linux} enthusiast with a strong passion for problem solving.
% My interests lie in solution development.
% From inception of ideas, to execution, and ultimately customer satisfaction.
% I enjoy working in teams driven by success, and I pursue it until we succeed.
\end{document}
